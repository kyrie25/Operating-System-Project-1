\subsubsection{Mô tả bài toán}
Chương trình \texttt{find} được thiết kế để tìm kiếm tất cả các tệp trong cây thư mục có tên trùng với tên cho trước. 
Chương trình nhận vào hai tham số: \textit{đường dẫn bắt đầu tìm kiếm} và \textit{tên file cần tìm}. 
Khi thực thi, chương trình sẽ duyệt qua toàn bộ thư mục con và in ra đường dẫn đầy đủ của các tệp phù hợp.

\subsubsection{Phương pháp thực hiện}
Dựa trên cấu trúc của chương trình \texttt{ls.c} trong hệ điều hành xv6, xây dựng hàm \texttt{find()} có khả năng duyệt thư mục theo cơ chế đệ quy. Chương trình được chia thành hai phần chính: hàm \texttt{find()} đảm nhiệm việc kiểm tra và xử lý từng đường dẫn, và hàm \texttt{main()} chịu trách nhiệm nhận tham số đầu vào cũng như gọi hàm tìm kiếm.
\paragraph{Hàm \texttt{void find(char *path, char *search)}}
Thực hiện việc mở đường dẫn, xác định loại đối tượng (file hoặc thư mục), và nếu là thư mục thì tiếp tục đọc các entry bên trong để gọi đệ quy tìm kiếm sâu hơn.
\begin{itemize}
    \item Đầu tiên, hàm mở đường dẫn được truyền vào bằng \texttt{open()} và lưu lại trong một file descriptor.
    \item Tiếp theo, sử dụng system call \texttt{fstat()} để lấy thông tin về đối tượng, nhằm xác định đó là file hay thư mục.
    \item Nếu là \textbf{file}, hàm so sánh phần tên cuối cùng trong đường dẫn với từ khóa cần tìm bằng \texttt{strcmp()}. Nếu trùng khớp, in ra đường dẫn của file.
    \item Nếu là \textbf{thư mục (T\_DIR)}, hàm đọc từng entry trong thư mục bằng \texttt{read()}, bỏ qua hai entry đặc biệt ``.'' và ``..''. Sau đó nối đường dẫn con và gọi lại chính hàm \texttt{find()} để tiếp tục tìm kiếm trong các thư mục con.
\end{itemize}

\paragraph{Hàm \texttt{int main(int argc, char *argv[])}}
Chịu trách nhiệm kiểm tra tham số đầu vào và gọi hàm \texttt{find()} để thực hiện việc tìm kiếm. 

\begin{itemize}
    \item Chương trình yêu cầu người dùng nhập đúng hai tham số: \textit{đường dẫn bắt đầu} và \textit{tên file cần tìm}. Nếu số lượng đối số nhỏ hơn 3 (tức là thiếu tham số), chương trình sẽ in thông báo hướng dẫn sử dụng qua \texttt{fprintf()} và kết thúc bằng \texttt{exit(1)}. 
    \item Khi nhận đủ đối số, hàm \texttt{find()} được gọi với \texttt{argv[1]} là đường dẫn và \texttt{argv[2]} là tên file cần tìm. Hàm \texttt{find()} trả về số lượng file được tìm thấy, và nếu giá trị này bằng 0, chương trình sẽ in thông báo “\textit{No files found matching 'filename'}” để cho biết không có kết quả nào trùng khớp. Cuối cùng, chương trình kết thúc bằng \texttt{exit(0)}.
\end{itemize}




